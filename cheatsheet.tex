\documentclass{article}
\usepackage[landscape]{geometry}
\usepackage{url}
\usepackage{multicol}
\usepackage{enumitem}
\usepackage{amsmath}
\usepackage{multirow}
\usepackage{esint}
\usepackage{amsfonts}
\usepackage{tikz}
\usetikzlibrary{decorations.pathmorphing}
\usepackage{amsmath,amssymb}

\usepackage{colortbl}
\usepackage{xcolor}
\usepackage{mathtools}
\usepackage{amsmath,amssymb}
\usepackage{enumitem}
\makeatletter

\newcommand*\bigcdot{\mathpalette\bigcdot@{.5}}
\newcommand*\bigcdot@[2]{\mathbin{\vcenter{\hbox{\scalebox{#2}{$\m@th#1\bullet$}}}}}
\makeatother

\title{130 Cheat Sheet}
\usepackage[brazilian]{babel}
\usepackage[utf8]{inputenc}

\advance\topmargin-.8in
\advance\textheight3in
\advance\textwidth3in
\advance\oddsidemargin-1.5in
\advance\evensidemargin-1.5in
\parindent0pt
\parskip2pt
\newcommand{\hr}{\centerline{\rule{3.5in}{1pt}}}
%\colorbox[HTML]{e4e4e4}{\makebox[\textwidth-2\fboxsep][l]{texto}
\begin{document}

\begin{center}{\huge{\textbf{Probability and Statistics for Economics Cheatsheet}}}\\
\end{center}
\begin{multicols*}{3}

\tikzstyle{mybox} = [draw=black, fill=white, very thick,
    rectangle, rounded corners, inner sep=5pt, inner ysep=10pt]
\tikzstyle{fancytitle} =[fill=black, text=white, font=\bfseries]

%------------ Random experiment (Lec1) ---------------
\begin{tikzpicture}
\node [mybox] (box){%
    \begin{minipage}{0.3\textwidth}
    The outcome in a random experiment is \textbf{unpredictable}:
    \scriptsize
    \begin{itemize}[topsep=0pt]
        \setlength\itemsep{-0.5em}
        \item[-] outcome is too complicates or poorly understood
        \item[-] outcome is designed to be unpredictable
        \item[-] coincidences, or independent chains of events
    \end{itemize}
    \footnotesize
    \textbf{Random card shuffle experiment}: take top card from a deck and insert randomly, to complete the shuffle of $n$ cards, we need $T=n+\frac{n}{2}+\cdots +\frac{n}{n-1}+1= n\log n$ shuffles.
    
    \textbf{Random number generator}: $x_{n+1}=\frac{ax_n+b}{c}-\left[\frac{ax_n+b}{c}\right]$, the remainder after dividing by $c$, hence $x_{n+1}\in [0,c-1]$, let $u_{n+1}=\frac{x_{n+1}}{c}$, $x_0,a,b,c$ all be integers. For very large $a$ and good choice of $b,c$, the sequence $u_1,u_2,\cdots$ is like a sequence of numbers randomly picked from $[0,1]$
    \end{minipage}
};
%------------ Random experiment (Lec 1) --------------
\node[fancytitle, right=10pt] at (box.north west) {Random experiments};
\end{tikzpicture}

%------------Probabilities (Lec 1) ---------------
\begin{tikzpicture}
\node [mybox] (box){%
    \begin{minipage}{0.3\textwidth}
    Ways of assigning probabilities:
    \scriptsize
    \begin{itemize}[topsep=0pt]
        \setlength\itemsep{-0.5em}
        \item[-] symmetry: assume all outcomes are equally likely
        \item[-] experimental method: relative frequency in repeated random experiment
        \item[-] subjective method: assign probabilities using knowledge of random experiment
        \item[-] market method
    \end{itemize}
    \normalsize
    Elements of \boxed{\textbf{probability space}}:
    \begin{enumerate}[noitemsep, topsep=0pt]
        \item[-] outcome space $\Omega$ and outcomes $\omega \in \Omega$
        \item[-] event $E$, $E\subset \Omega$
        \item[-] probability function/measure $P:\mathcal{A}\rightarrow [0,1]$: a function from \textbf{a collection} $\mathcal{A}$ of subsets of $\Omega$ to the interval $[0,1]$.
    \end{enumerate}
    \end{minipage}
};
%------------ Mixing Header ---------------------
\node[fancytitle, right=10pt] at (box.north west) {Probabilities};
\end{tikzpicture}

%------------ Events (Lec 1) ---------------
\begin{tikzpicture}
\node [mybox] (box){%
    \begin{minipage}{0.3\textwidth}
    Events $E_1,E_2,\cdots$ are just sets. They also follow the algebras of sets, such as:
    
    \scriptsize{
    	\begin{tabular}{ll}
        $A\cup (B\cap C)=(A\cup B)\cap (A\cup C)$ \\
        $A\cap (B\cup C)=(A\cap B)\cup (A\cap C)$ \\
        $A\cup A^C=U$ & $ A\setminus B=A\cap B^C$ \\
        $(\bigcup_{i=1}^{\infty}A_i)^C=\bigcap_{i=1}^{\infty}A_i^C$ & $(\bigcap_{i=1}^{\infty}A_i)^C=\bigcup_{i=1}^{\infty}A_i^C$ \\
        $A\cup B=U,A\cap B=\varnothing\Leftrightarrow B=A^C$ &$(A^C)^C=A$ \\
        \multicolumn{2}{l}{$A\subseteq B\Leftrightarrow A\cap B=A\Leftrightarrow A\cup B=B\Leftrightarrow A\setminus B=\varnothing \Leftrightarrow B^C\subseteq A^C$} 
        \end{tabular}}
    \normalsize
    
    Two special relations:
    \begin{enumerate}[noitemsep, topsep=0pt]
        \item[-] \textbf{disjoint}: $E_1\cap E_2=\varnothing$
        \item[-] \textbf{partition}: $\bigcup^{\infty}_{i=1}E_i=\Omega$
    \end{enumerate}
    \end{minipage}
};
%------------ Inner Product Space Header ---------------------
\node[fancytitle, right=10pt] at (box.north west) {Classes of events};
\end{tikzpicture}

%------------ sigma field (Lec1) ---------------
\begin{tikzpicture}
\node [mybox] (box){%
    \begin{minipage}{0.3\textwidth}
    Definition of $\sigma-$field: $\mathcal{A}$ (a collection of subsets of $\Omega$) is a $\sigma-$field if
    \begin{itemize}[noitemsep, topsep=0pt]
        \item[(i)] $\varnothing \in \mathcal{A}$
        \item[(ii)] $E\in \mathcal{A}\Rightarrow E^C\in \mathcal{A}$
        \item[(iii)] $E_1,E_2,\cdots \in \mathcal{A}\Rightarrow \bigcup^{\infty}_{i=1}E_i\in\mathcal{A}$
    \end{itemize}
    
    smallest $\sigma-$field:$\mathcal{A}=\{\varnothing,\Omega\}$
    \end{minipage}
};
%------------ sigma field (Lec1) ---------------------
\node[fancytitle, right=10pt] at (box.north west) {$\sigma-$field and Borel $\sigma-$field};
\end{tikzpicture}

%------------ Variation of Parameters Content ---------------------
\begin{tikzpicture}
\node [mybox] (box){%
    \begin{minipage}{0.3\textwidth}
    	\begin{align*}
        	F(x) &= y'' + y' \\
            y_h &= b_1y_1(x) + b_2y_2(x), y_1 y_2 \text{ are L.I.} \\
            y_p &= u_1(x)y_1(x) + u_2(x)y_2(x) \\
            u_1 &= \int^t -\frac{y_2F(t)dt}{w[y_1,y_2](t)} \\
            u_2 &= \int^t \frac{y_1F(t)dt}{w[y_1,y_2](t)} \\    		
            y &= y_h + y_p
    	\end{align*}
    	
    \end{minipage}
};
%------------ Variation of Parameters Header ---------------------
\node[fancytitle, right=10pt] at (box.north west) {Variation of Parameters};
\end{tikzpicture}

%------------ Systems of ODE Content ---------------
\begin{tikzpicture}
\node [mybox] (box){%
    \begin{minipage}{0.3\textwidth}
    \small{
    	\begin{tabular}{lp{4cm} l}
        $\vec{x}' = A\vec{x}$ \\
		\textit{A is diagonalizable} & $\vec{x}(t)=a_{1}e^{\lambda_1 t}\vec{v_1}+\cdots+ a_{n}e^{\lambda_n t}\vec{v_n}$ \\ \hline
        \textit{A is not diagonalizable} & $\vec{x}(t)=a_1e^{\lambda_1 t}\vec{v_1} + a_2e^{\lambda t}(\vec{w} + t\vec{v} )$ \\
        & where $(A - \lambda I)\vec{w} = \vec{v} $\\
        & $\vec{v}$ is an Eigenvector w/ value $\lambda$ \\
        & i.e. $\vec{w}$ is a generalized Eigenvector \\ \hline
        $\vec{x}' = A\vec{x} + \vec{B}$ &Solve $y_h$ \\
        & $\vec{x_1} = e^{\lambda_1t}\vec{v_1}, \vec{x_2} = e^{\lambda_2t}\vec{v_2}$ \\ 			& $\vec{X} = [\vec{x_1},\vec{x_2}]$ \\
        & $\vec{X}\vec{u}'=\vec{B}$ \\
        & $y_p = \vec{X}\vec{u}$ \\
        & $y = y_h + y_p$
	\end{tabular}}
    \end{minipage}
};
%------------ Systems of ODE Header ---------------------
\node[fancytitle, right=10pt] at (box.north west) {Systems};
\end{tikzpicture}

%------------ Exponentiation Content ---------------
\begin{tikzpicture}
\node [mybox] (box){%
    \begin{minipage}{0.3\textwidth}
    \small{
    	\begin{tabular}{lp{4cm} l}
        $A^n = SD^nS^{-1}$ \\
        \textit{D is the diagonalization of A}
	\end{tabular}}
    \end{minipage}
};
%------------ Spring-Mass Header ---------------------
\node[fancytitle, right=10pt] at (box.north west) {Matrix Exponentiation};
\end{tikzpicture}

%------------ Laplace Transforms Content ---------------
\begin{tikzpicture}
\node [mybox] (box){%
    \begin{minipage}{0.3\textwidth}
    $$L[f](s) = \int_0^{\infty} e^{-sx}f(x)dx $$
    
    \small{
    	\begin{tabular}{lp{4cm} l}
        $f(t) = t^n, n \geq 0 $ &$F(s) = \frac{n!}{s^{n+1}}, s > 0 $ \\
        $f(t) = e^{at}, a \textit{ constant}$ & $ F(s) = \frac{1}{s-a}, s > a$ \\
        $f(t) = \sin{bt}, b \textit{ constant}$ & $ F(s) = \frac{b}{s^2 + b^2}, s > 0$ \\
        $f(t) = \cos{bt}, b \textit{ constant}$ & $ F(s) = \frac{s}{s^2 + b^2}, s > 0$ \\
        $f(t) = t^{-1/2}$ & $F(s) = \frac{\pi}{s^{1/2}}, s > 0$ \\
        $f(t) = \delta(t-a)$ & $F(s) = e^{-as}$ \\
        $f'$ & $L[f'] = sL[f] - f(0)$ \\
        $f''$ & $L[f''] = s^2 L[f] - sf(0) - f'(0)$ \\
        $L[e^{at}f(t)]$ & $L[f](s-a)$ \\
        $L[u_a(t)f(t-a)]$ & $L[f]e^{-as}$ 
        \end{tabular}}
    \end{minipage}
};
%------------ Laplace Transforms Header ---------------------
\node[fancytitle, right=10pt] at (box.north west) {Laplace Transforms};
\end{tikzpicture}
%------------ Gaussian Integral Content ---------------------
\begin{tikzpicture}
\node [mybox] (box){%
    \begin{minipage}{0.3\textwidth}
	$\int_{-\infty}^{+\infty} e^{-1/2(\vec{x}^TA\vec{x})} = \frac{\sqrt{2\pi}^n}{\sqrt{\det A}}$
	\end{minipage}
};
%------------ Gaussian Integral Header ---------------------
\node[fancytitle, right=10pt] at (box.north west) {Gaussian Integral};
\end{tikzpicture}
\\
\\
\\
\\

%------------ Complex Numbers Content ---------------------
\begin{tikzpicture}
\node [mybox] (box){%
    \begin{minipage}{0.3\textwidth}
    \small{
        	\begin{tabular}{lp{4cm} l}
            \textit{Systems of equations} & If $\vec{w_1} = \vec{u(t)} + i\vec{v(t)}$ is a solution, $\vec{x_1} = \vec{u(t)}, \vec{x_2} = \vec{v(t)}$ are solutions \\ 
            & i.e. $\vec{x_h} = c_1 \vec{x_1} + c_2 \vec{x_2}$ \\
            \hline
            \textit{Euler's Identity} &$e^{ix} = \cos x + i \sin x$
			\end{tabular}
    }
	\end{minipage}
};
%------------ Gaussian Integral Header ---------------------
\node[fancytitle, right=10pt] at (box.north west) {Complex Numbers};
\end{tikzpicture}

%------------ Vector Spaces ---------------
\begin{tikzpicture}
\node [mybox] (box){%
    \begin{minipage}{0.3\textwidth}
    $v_1, v_2 \in V$\\
    1. $v_1 + v_2 \in V$ \\
	2. $k \in \mathbb{F}, kv_1 \in V $ \\
	3. $ v_1 + v_2 = v_2 + v_1 $ \\
	4. $(v_1 + v_2) + v_3 = v_1 + (v_2 + v_3) $ \\
	5. $\forall v \in V, 0 \in V \mid 0 + v_1 = v_1 + 0 = v_1$ \\
    6. $\forall v \in V, \exists -v \in V \mid v + (-v) = (-v) + v = 0 $ \\
    7. $\forall v \in V, 1 \in \mathbb{F} \mid 1*v = v$ \\
    8. $\forall v \in V, k,l \in \mathbb{F}, (kl)v = k (lv)$ \\
    9. $\forall k \in \mathbb{F}, k(v_1 + v_2) = kv_1 + kv_2$ \\
    10. $\forall v \in V, k,l \in \mathbb{F}, (k+l)v = kv + lv$
    \end{minipage}
};
%------------ Vector Space Header ---------------------
\node[fancytitle, right=10pt] at (box.north west) {Vector Spaces};
\end{tikzpicture}
\end{multicols*}
\end{document}


Contact GitHub API Training Shop Blog About
© 2016 GitHub, Inc. Terms Privacy Security Status Help