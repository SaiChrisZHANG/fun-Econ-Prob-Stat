\documentclass[10pt,landscape,a4paper]{article}
\usepackage[utf8]{inputenc}
\usepackage[ngerman]{babel}
\usepackage{tikz}
\usetikzlibrary{shapes,positioning,arrows,fit,calc,graphs,graphs.standard}
\usepackage[nosf]{kpfonts}
\usepackage[t1]{sourcesanspro}
%\usepackage[lf]{MyriadPro}
%\usepackage[lf,minionint]{MinionPro}
\usepackage{multicol}
\usepackage{wrapfig}
\usepackage[top=2mm,bottom=2mm,left=2mm,right=1mm]{geometry}
\usepackage[framemethod=tikz]{mdframed}
\usepackage{microtype}
\usepackage{hyperref}

\usepackage{url}
\usepackage{multirow}
\usepackage{esint}
\usepackage{amsfonts}
\usetikzlibrary{decorations.pathmorphing}

\usepackage{colortbl}
\usepackage{xcolor}
\usepackage{mathtools}
\usepackage{amsmath,amssymb}
\usepackage{enumitem}
\makeatletter

\let\bar\overline

\setlist[itemize]{topsep=0pt,leftmargin=15pt,itemsep=-0.2em}
\definecolor{myblue}{cmyk}{1,.72,0,.38}
\definecolor{mypurple}{cmyk}{.57,1,0,.58}
\definecolor{myred}{cmyk}{0,.88,.88,.58}
\definecolor{mygreen}{cmyk}{1,0,.69,.66}
\definecolor{myorange}{cmyk}{0,.58,100,.20}

\def\firstcircle{(0,0) circle (1.5cm)}
\def\secondcircle{(0:2cm) circle (1.5cm)}

\pgfdeclarelayer{background}
\pgfsetlayers{background,main}

\renewcommand{\baselinestretch}{.8}
\pagestyle{empty}

\global\mdfdefinestyle{header}{%
linecolor=gray,linewidth=1pt,%
leftmargin=0mm,rightmargin=0mm,skipbelow=0mm,skipabove=0mm,
}

%\newcommand{\header}{
%\begin{mdframed}[style=header]
%\footnotesize
%\sffamily
%Cheat sheet\\
%by~Your~Name,~page~\thepage~of~2
%\end{mdframed}
%}

\makeatletter
\renewcommand{\section}{\@startsection{section}{1}{0mm}{1ex}{.2ex}{\normalsize\bfseries}}
\renewcommand{\subsection}{\@startsection{subsection}{1}{0mm}{.2ex}{.2ex}{\bfseries}}

\newcommand*\bigcdot{\mathpalette\bigcdot@{.5}}
\newcommand*\bigcdot@[2]{\mathbin{\vcenter{\hbox{\scalebox{#2}{$\m@th#1\bullet$}}}}}
\makeatother

\def\multi@column@out{%
   \ifnum\outputpenalty <-\@M
   \speci@ls \else
   \ifvoid\colbreak@box\else
     \mult@info\@ne{Re-adding forced
               break(s) for splitting}%
     \setbox\@cclv\vbox{%
        \unvbox\colbreak@box
        \penalty-\@Mv\unvbox\@cclv}%
   \fi
   \splittopskip\topskip
   \splitmaxdepth\maxdepth
   \dimen@\@colroom
   \divide\skip\footins\col@number
   \ifvoid\footins \else
      \leave@mult@footins
   \fi
   \let\ifshr@kingsaved\ifshr@king
   \ifvbox \@kludgeins
     \advance \dimen@ -\ht\@kludgeins
     \ifdim \wd\@kludgeins>\z@
        \shr@nkingtrue
     \fi
   \fi
   \process@cols\mult@gfirstbox{%
%%%%% START CHANGE
\ifnum\count@=\numexpr\mult@rightbox+2\relax
          \setbox\count@\vsplit\@cclv to \dimexpr \dimen@-1cm\relax
\setbox\count@\vbox to \dimen@{\vbox to 1cm{\header}\unvbox\count@\vss}%
\else
      \setbox\count@\vsplit\@cclv to \dimen@
\fi
%%%%% END CHANGE
            \set@keptmarks
            \setbox\count@
                 \vbox to\dimen@
                  {\unvbox\count@
                   \remove@discardable@items
                   \ifshr@nking\vfill\fi}%
           }%
   \setbox\mult@rightbox
       \vsplit\@cclv to\dimen@
   \set@keptmarks
   \setbox\mult@rightbox\vbox to\dimen@
          {\unvbox\mult@rightbox
           \remove@discardable@items
           \ifshr@nking\vfill\fi}%
   \let\ifshr@king\ifshr@kingsaved
   \ifvoid\@cclv \else
       \unvbox\@cclv
       \ifnum\outputpenalty=\@M
       \else
          \penalty\outputpenalty
       \fi
       \ifvoid\footins\else
         \PackageWarning{multicol}%
          {I moved some lines to
           the next page.\MessageBreak
           Footnotes on page
           \thepage\space might be wrong}%
       \fi
       \ifnum \c@tracingmulticols>\thr@@
                    \hrule\allowbreak \fi
   \fi
   \ifx\@empty\kept@firstmark
      \let\firstmark\kept@topmark
      \let\botmark\kept@topmark
   \else
      \let\firstmark\kept@firstmark
      \let\botmark\kept@botmark
   \fi
   \let\topmark\kept@topmark
   \mult@info\tw@
        {Use kept top mark:\MessageBreak
          \meaning\kept@topmark
         \MessageBreak
         Use kept first mark:\MessageBreak
          \meaning\kept@firstmark
        \MessageBreak
         Use kept bot mark:\MessageBreak
          \meaning\kept@botmark
        \MessageBreak
         Produce first mark:\MessageBreak
          \meaning\firstmark
        \MessageBreak
        Produce bot mark:\MessageBreak
          \meaning\botmark
         \@gobbletwo}%
   \setbox\@cclv\vbox{\unvbox\partial@page
                      \page@sofar}%
   \@makecol\@outputpage
     \global\let\kept@topmark\botmark
     \global\let\kept@firstmark\@empty
     \global\let\kept@botmark\@empty
     \mult@info\tw@
        {(Re)Init top mark:\MessageBreak
         \meaning\kept@topmark
         \@gobbletwo}%
   \global\@colroom\@colht
   \global \@mparbottom \z@
   \process@deferreds
   \@whilesw\if@fcolmade\fi{\@outputpage
      \global\@colroom\@colht
      \process@deferreds}%
   \mult@info\@ne
     {Colroom:\MessageBreak
      \the\@colht\space
              after float space removed
              = \the\@colroom \@gobble}%
    \set@mult@vsize \global
  \fi}

\hypersetup{
    colorlinks=true,
    linkcolor=myblue,
    filecolor=magenta,      
    urlcolor=myblue,
    pdfpagemode=FullScreen,
    }

\urlstyle{same}

\makeatother
\setlength{\parindent}{0pt}

% material references: Prof. Geert Ridder's lecture notes
% latex coding references: https://github.com/tim-st/latex-cheatsheet, https://www.overleaf.com/latex/templates/hoja-de-ecuaciones-electricidad-y-magnetismo/xwgjqkrjjgcb

\begin{document}
\begin{center}{\large{\textbf{Probability and Statistics for Economics Cheat Sheet}}}\\
Author: Sai Zhang (\href{mailto:saizhang.econ@gmail.com}{email} me or check my \href{https://github.com/SaiChrisZHANG}{Github} page)
\end{center}

\small
\begin{multicols*}{5}

% set box styles: in this file, I only use red blue and purple boxes
%% blue boxes
\tikzstyle{bluebox} = [draw=myblue, fill=white, thick, rectangle, rounded corners, inner sep=5pt, inner ysep=10pt, text=myblue]
\tikzstyle{bluetitle} =[fill=myblue, text=white, font=\bfseries]
\tikzstyle{ibluebox} = [draw=myblue, fill=myblue, thick, rectangle, rounded corners, inner sep=5pt, inner ysep=10pt, text=white]
\tikzstyle{ibluetitle} =[draw=myblue, fill=white, text=myblue, font=\bfseries]
%% red boxes
\tikzstyle{redbox} = [draw=myred, fill=white, thick, rectangle, rounded corners, inner sep=5pt, inner ysep=10pt, text=myred]
\tikzstyle{redtitle} =[fill=myred, text=white, font=\bfseries]
\tikzstyle{iredbox} = [draw=myred, fill=myred, thick, rectangle, rounded corners, inner sep=5pt, inner ysep=10pt, text=white]
\tikzstyle{iredtitle} =[draw=myred, fill=white, text=myred, font=\bfseries]
%% purple boxes
\tikzstyle{purplebox} = [draw=mypurple, fill=white, thick, rectangle, rounded corners, inner sep=5pt, inner ysep=10pt, text=mypurple]
\tikzstyle{purpletitle} =[fill=mypurple, text=white, font=\bfseries]
\tikzstyle{ipurplebox} = [draw=mypurple, fill=mypurple, thick, rectangle, rounded corners, inner sep=5pt, inner ysep=10pt, text=white]
\tikzstyle{ipurpletitle} =[draw=mypurple, fill=white, text=mypurple, font=\bfseries]
%% orange boxes
\tikzstyle{orangebox} = [draw=myorange, fill=white, thick, rectangle, rounded corners, inner sep=5pt, inner ysep=10pt, text=myorange]
\tikzstyle{orangetitle} =[fill=myorange, text=white, font=\bfseries]
\tikzstyle{iorangebox} = [draw=myorange, fill=myorange, thick, rectangle, rounded corners, inner sep=5pt, inner ysep=10pt, text=white]
\tikzstyle{iorangetitle} =[draw=myorange, fill=white, text=myorange, font=\bfseries]

\section*{Random experiments}
The outcome in a random experiment is \textbf{unpredictable}:
\begin{itemize}
    \item[-] outcome is too complicates or poorly understood
    \item[-] outcome is designed to be unpredictable
    \item[-] coincidences, or independent chains of events
\end{itemize}

\vspace{5pt}
\begin{tikzpicture}
\node [orangebox] (box){%
    \begin{minipage}{0.18\textwidth}
    \footnotesize
    \textbf{Random card shuffle experiment}: take top card from a deck and insert randomly, to complete the shuffle of $n$ cards, we need $$T=n+\frac{n}{2}+\cdots +\frac{n}{n-1}+1= n\log n$$ shuffles.
    
    \textbf{Random number generator}: $$x_{n+1}=\frac{ax_n+b}{c}-\left[\frac{ax_n+b}{c}\right]$$the remainder after dividing by $c$, hence $x_{n+1}\in [0,c-1]$, let $u_{n+1}=\frac{x_{n+1}}{c}$, $x_0,a,b,c$ all be integers. For very large $a$ and good choice of $b,c$, the sequence $u_1,u_2,\cdots$ is like a sequence of numbers randomly picked from $[0,1]$
    \end{minipage}
};
\node[orangetitle, right=4pt] at (box.north west) {Two examples};
\end{tikzpicture}

\section*{Probabilities}
Probability is a number in $[0,1]$ that measures the likelihood of an outcome or a set of outcomes.

Ways of assigning probabilities:
\begin{itemize}
    \item[-] \textbf{symmetry}: assume all outcomes are equally likely
    \item[-] \textbf{experimental method}: relative frequency in repeated random experiment
    \item[-] \textbf{subjective method}: assign probabilities using knowledge of random experiment
    \item[-] \textbf{market method}
\end{itemize}

\vspace{5pt}
\begin{tikzpicture}
\node [bluebox] (box){%
    \begin{minipage}{0.18\textwidth}
     \begin{itemize}
        \item[-] \textbf{outcome space} $\Omega$ and outcomes $\omega \in \Omega$
        \item[-] \textbf{event} $E$, $E\subset \Omega$
        \item[-] \textbf{probability function/measure} $P:\mathcal{A}\rightarrow [0,1]$: a function from \textbf{a collection} $\mathcal{A}$ of subsets of $\Omega$ to the interval $\mathbf{[0,1]}$.
    \end{itemize}
    \end{minipage}
};
\node[bluetitle, right=4pt] at (box.north west) {Elements of probability space};
\end{tikzpicture}

\section*{Classes of events}
Events $E_1,E_2,\cdots$ are just sets. They also follow the algebras of sets.

\begin{tikzpicture}
\node [orangebox] (box){%
    \begin{minipage}{0.18\textwidth}
    \footnotesize
    $A\cup (B\cap C)=(A\cup B)\cap (A\cup C)$
    
    $A\cap (B\cup C)=(A\cap B)\cup (A\cap C)$
    
    $A\cup A^C=U$, $ A\setminus B=A\cap B^C$
    
    $(\bigcup_{i=1}^{\infty}A_i)^C=\bigcap_{i=1}^{\infty}A_i^C$
    
    $(\bigcap_{i=1}^{\infty}A_i)^C=\bigcup_{i=1}^{\infty}A_i^C$
    
    $A\cup B=U,A\cap B=\varnothing\Leftrightarrow B=A^C$
    
    $(A^C)^C=A$
    
    $A\subseteq B\Leftrightarrow A\cap B=A\Leftrightarrow A\cup B=B\Leftrightarrow A\setminus B=\varnothing \Leftrightarrow B^C\subseteq A^C$
    \end{minipage}
};
\node[orangetitle, right=4pt] at (box.north west) {Some set algebras};
\end{tikzpicture}

Two special relations:
\begin{itemize}
    \item[-] \textbf{disjoint}: $E_1\cap E_2=\varnothing$
    \item[-] \textbf{partition}: $\bigcup^{\infty}_{i=1}E_i=\Omega$, $\{E_i\}$ are pairwise disjoint
\end{itemize}

\section*{$\sigma-$field and Borel $\sigma-$field}

\vspace{5pt}
\begin{tikzpicture}
\node [bluebox] (box){%
    \begin{minipage}{0.18\textwidth}
     $\mathcal{A}$ (a collection of subsets of $\Omega$) is a $\sigma-$field if:
\begin{itemize}
    \item[\textbf{1}] $\varnothing \in \mathcal{A}$
    \item[\textbf{2}] $E\in \mathcal{A}\Rightarrow E^C\in \mathcal{A}$
    \item[\textbf{3}] $E_1,E_2,\cdots \in \mathcal{A}\Rightarrow \bigcup^{\infty}_{i=1}E_i\in\mathcal{A}$
\end{itemize}
It is easy to see that $\bigcap E^C_i \in\mathcal{A}$, $\bigcup E^C_i\in\mathcal{A}$, $\bigcap E_i\in\mathcal{A}$ as well
    \end{minipage}
};
\node[bluetitle, right=4pt] at (box.north west) {Definition of $\sigma-$field};
\end{tikzpicture}
    
Two important $\sigma-$field:
\begin{itemize}
    \item[-] Trivial $\sigma-$field:$\mathcal{A}=\{\varnothing,\Omega\}$
    \item[-] Largest $\sigma-$field: \textbf{powerset} of $\Omega$, $\mathcal{P}(\Omega)$
\end{itemize}

\end{multicols*}

\end{document}